\documentclass{Metanorma}

\title{Sample document}
\set{author}{John Doe}
\set{author-phonetic}{jän dō}

\begin{document}
  \maketitle
  \tableofcontents
  \bigskip

  The author of this document is \get{author}, which is pronounced
  \get{author-phonetic}.\\ This document contains random snippets
  from \cite{ISO7301} and is written using \LaTeX --
  details about that in \cite{knuthwebsite}.

  %%%%%%%%%%%%%%%%%%%%%%%%%%%%%%%%%%%%%%%%%%%%%%%%%%%%%%%%%%%%%%%%%%%%%%%%%%%%%%%%

  \section{Introduction}

  Here is a paragraph that contains formatting samples: \textbf{bold},
  \textit{italic}, \textsc{small caps} and \textst{strikethrough}
  are all available.

  \begin{note}
    The next few paragraphs are explicitly aligned.
  \end{note}

  \begin{flushleft}
    Paragraphs can be aligned left.
  \end{flushleft}

  \begin{center}
    Paragraphs can be centered.
  \end{center}

  \begin{flushright}
    Paragraphs can be aligned right.
  \end{flushright}

  \begin{caution}
    The upcoming table contains possibly inaccurate informations.
  \end{caution}

  \begin{table}\label{tab:example}
    \caption{This week's weather}
    \begin{tabular}{ll}
      Day & Summary \\
      Monday & A clear day with lots of sunshine.\\
      Tuesday & Cloudy with rain, across many northern regions.\\
      Wednesday & Rain will still linger for the morning.\\
    \end{tabular}
  \end{table}

  \begin{tip}
    \LaTeX is very convenient when typesetting high-quality documents;
    you can easily insert captioned frames
    like table \ref{tab:example} and figure \ref{fig:example}.
  \end{tip}

  \begin{figure}\label{fig:example}
    \centering
    \caption{Metanorma's logo and payoff.}
    \includegraphics[width=2in]{metanorma.png}
  \end{figure}

  %%%%%%%%%%%%%%%%%%%%%%%%%%%%%%%%%%%%%%%%%%%%%%%%%%%%%%%%%%%%%%%%%%%%%%%%%%%%%%%%

  \section{Terms and Definitions}

  \subsection{paddy}
  \label{paddy}
  \alt{paddy rice}
  \alt{rough rice}
  \deprecated{cargo rice}
  \domain{rice}

  rice retaining its husk after threshing

  \begin{example}
    Foreign seeds, husks, bran, sand, dust.
  \end{example}

  \begin{note}
    The starch of waxy rice consists almost entirely of amylopectin.
    The kernels have a tendency to stick together after cooking.
  \end{note}

  \begin{source}
    \cite{ISO7301}, The term "cargo rice" is shown as deprecated,
    and Note 1 to entry is not included here
  \end{source}

  %%%%%%%%%%%%%%%%%%%%%%%%%%%%%%%%%%%%%%%%%%%%%%%%%%%%%%%%%%%%%%%%%%%%%%%%%%%%%%%%

  \section{Symbols and Abbreviations}

  \begin{description}
    \item[CPU] The brain of the computer.
    \item[Hard drive] Permanent storage for operating system and/or user files.
    \item[RAM] Temporarily stores information the CPU uses during operation.
    \item[Keyboard] Used to enter text or control items on the screen.
    \item[Mouse] Used to point to and select items on your computer screen.
    \item[Monitor] Displays information in visual form using text and graphics.
  \end{description}

  %%%%%%%%%%%%%%%%%%%%%%%%%%%%%%%%%%%%%%%%%%%%%%%%%%%%%%%%%%%%%%%%%%%%%%%%%%%%%%%%

  \begin{thebibliography}{2}
    \bibitem[ISO 7301]{ISO7301} Rice -- Specification

    \bibitem{knuthwebsite}
      Knuth: Computers and Typesetting,
      \\\texttt{http://www-cs-faculty.stanford.edu/\~{}uno/abcde.html}
  \end{thebibliography}
\end{document}
