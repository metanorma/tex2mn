\documentclass{Metanorma}

\begin{document}

  \section{Custom terms and definitions}
  \mn{heading=terms and definitions}

  Lipsum

  \section{Section}

  Lipsum

  \subsection{Subsection}
  \mn{heading=terms and definitions}

  Lipsum

  \section{Annex}
  \mn{annex}

  Lipsum

  \subsection{}

  Lipsum

  \subsubsection{Subsubsection}
  \mn{\%inline-header}

  Lipsum

  \paragraph{Paragraph}

  Lipsum

  \subparagraph{Subparagraph}

  Lipsum

  \subparagraph{Subparagraph l6}
  \mn{level=6}

  Lipsum

  \subparagraph{Subparagraph l7}
  \mn{level=7}

  Lipsum

  \section{Tests for text formatting}

  Bold \textbf{bold} and {\bfseries bold group}.

  Emphasis \textit{emphasis} and {\itshape emphasis group}.

  Small caps \textsc{small caps} and {\scshape small caps group}.

  Strikethrough \textst{strikethrough}.

  \section{Tests for paragraph alignment}

  This paragraph is justified.
  This paragraph is justified.
  This paragraph is justified.
  This paragraph is justified.
  This paragraph is justified.

  This one too. This one too. This one too.
  This one too. This one too. This one too.
  This one too. This one too. This one too.
  This one too. This one too. This one too.
  This one too. This one too. This one too.

  \begin{flushleft}
    This paragraph is left aligned.
    This paragraph is left aligned.
    This paragraph is left aligned.
    This paragraph is left aligned.
    This paragraph is left aligned.

    This one too. This one too. This one too.
    This one too. This one too. This one too.
    This one too. This one too. This one too.
    This one too. This one too. This one too.
    This one too. This one too. This one too.
  \end{flushleft}

  \begin{center}
    This paragraph is centered.
    This paragraph is centered.
    This paragraph is centered.
    This paragraph is centered.
    This paragraph is centered.

    This one too. This one too. This one too.
    This one too. This one too. This one too.
    This one too. This one too. This one too.
    This one too. This one too. This one too.
    This one too. This one too. This one too.
  \end{center}

  \begin{flushright}
    This paragraph is right aligned.
    This paragraph is right aligned.
    This paragraph is right aligned.
    This paragraph is right aligned.
    This paragraph is right aligned.

    This one too. This one too. This one too.
    This one too. This one too. This one too.
    This one too. This one too. This one too.
    This one too. This one too. This one too.
    This one too. This one too. This one too.
  \end{flushright}

  \section{Tests for quotes}

  \begin{quote}
    This is the simples quote.
  \end{quote}

  \begin{quote}
    This is a quote with multiple paragraphs.

    Here is the second one.
  \end{quote}

  \begin{quote}
    \mn{ISO, "ISO7301,section 1"}
    This is a pretty complex quote.

    There could not be more stuff.

    In fact, there are attributes too!
  \end{quote}

  % \section*{Foreword}

  % This is the foreword.

  % And another one line.

  % \section*{Introduction}

  % This is the introduction.

  % And another one line.

  % \subsection*{Patent notice}
  % \lxRDFa[.]{adoc={yada yada,obligation=informative}}

  % This is the patent notice.

  % \subsection*{}

  % And another one line.

  % \section{Terms and definitions}

  % Yada yada.

  % \subsection{paddy}
  % \lxRDFa[.]{alternate={paddy rice,rough rice},deprecated=old rice,domain=rice}

  % Here are contents.

  % And another one line.

  % \section{Contents}
  % \lxRDFa[.]{adoc={obligation=normative}}

  % Here are contents.

  % And another one line.

\end{document}
